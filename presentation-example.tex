\documentclass[aspectratio=169]{beamer}
\usetheme{suaitex}
% собирать через xelatex (2 раза)!!!


% \usepackage{pgfpages} 
% \pgfpagesuselayout{resize to}[a4paper,border shrink=5mm,landscape]

\title{Тема презентации}
\date[ГУАП \the\year{}]{Подзаголовок (дата) -- \the\year{}}
\author[Автор]{Автор}
\supervisor{Руководитель}

\begin{document}

\begin{frame}[plain]
    \titlepage
\end{frame}


\section*{Outline}

\begin{frame} 
    \frametitle{Outline} 
    \tableofcontents 
\end{frame}


\section{Motivation}

\begin{frame}[t]
    \frametitle{What Are Prime Numbers?} 
    \begin{definition} 
        A \alert{prime number} is a number that has exactly two divisors. 
    \end{definition} 

    \begin{example} 
        \begin{itemize} 
        \item2 is prime(two divisors: 1 and 2). 
        \item3 is prime(two divisors: 1 and 3). 
        \item4 is not prime(\alert{three} divisors: 1, 2, and 4). 
        \end{itemize} 
    \end{example} 
\end{frame}

\begin{frame}
    \frametitle{What’s Still To Do?} 
    \begin{columns}[t]
        \column{0.45\textwidth}
        \begin{block}{Answered Questions} 
            How many primes are there? 
        \end{block} 
        \column{0.45\textwidth}
        \begin{block}{Open Questions} 
            Is every even number the sum of two primes? 
        \end{block} 
    \end{columns}
\end{frame}

\begin{frame}[fragile=singleslide]
    \frametitle{An Algorithm for Finding Prime Numbers}

    \begin{verbatim}
int main (void) {
    std::vector<bool> is_prime (100, true);
    for (int i = 2; i < 100; i++)
        if(is_prime[i]){
            std::cout << i << " ";
            for (int j = i; j < 100; is_prime[j] = false, j+=i);
        }
    return 0;
}
    \end{verbatim}

    \begin{uncoverenv}<2>
        Note the use of \verb|std::|.        
    \end{uncoverenv}
    
\end{frame}


\section{Theorem}

\begin{frame} 
    \frametitle{There Is No Largest Prime Number} 
    \begin{theorem}
    There is no largest prime number. \end{theorem} 
    \begin{itemize} 
        \item Suppose $p$ were the largest prime number. 
        \item Let $q$ be the product of the first $p$ numbers. 
        \item Then $q+1$ is not divisible by any of them. 
        \item But $q + 1$ is greater than $1$, thus divisible by some prime
        number not in the first $p$ numbers.
    \end{itemize}
\end{frame}

\begin{frame} 
    \frametitle{There Is No Largest Prime Number} 
    \framesubtitle{The proof uses \textit{reductio ad absurdum}.} 
    \begin{theorem}
    There is no largest prime number. \end{theorem} 
    \begin{enumerate} 
    \item<1-| alert@1> Suppose $p$ were the largest prime number. 
    \item<2-> Let $q$ be the product of the first $p$ numbers. 
    \item<2-> Then $q+1$ is not divisible by any of them. 
    \item<1-> But $q + 1$ is greater than $1$, thus divisible by some prime
    number not in the first $p$ numbers.
    \end{enumerate}
\end{frame}
    
\end{document}